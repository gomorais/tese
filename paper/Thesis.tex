\documentclass[11pt]{report}
\usepackage{graphicx}


\graphicspath{ {/home/morais/Documents/tese/writing/} }

%Gummi|061|=)
\title{\textbf{Measuring the effects of IPv4 address exhaustion on allocation and routing dynamics}}
\author{Goncalo Morais}
\date{}
\begin{document}
\maketitle
\section{Abstract}
In order to achieve connectivity in the Internet, the uniqueness of
globally routed IP address blocks must be ensured. BGP, the de facto
standard protocol used to enable connectivity between networks does not
provide mechanisms for resource certification, e.g., to enforce the
usage of certain IP address blocks only by their respective holder.
Hence, a body of largely decoupled registry mechanisms has evolved over
the years, aiming at providing accurate address-block registration data
to network operators.

The fact that we face IPv4 address exhaustion makes IPv4 addresses an
increasingly scarce resource. Hence, governance and control over IP
address block usage becomes an even more critical issue. We propose to
investigate the various mechanisms that are used in today's Internet
such as allocation registries provided by Regional Internet Registries
(RIRs) and routing registries provided by the Internet Routing Registry
(IRR) and correlate these databases with actual routing entries seen in
BGP data. We intend to asses how accurate this data really is and how it
evolved over time. Hereby we pay particular attention to scenarios
resulting out of IPv4 address exhaustion, such as possible address space
transfers.

\section{Introduction}
The Internet Registry System and IANA http://www.ripe.net/internet-coordination/internet-governance/internet-technical-community/the-rir-system and https://www.iana.org/numbers

The Internet Assigned Numbers Authority, also known as IANA, is responsible for the global coordination of the Internet Protocol (IP) address system and also of the Autonomous Systems Numbers which are used for routing traffic on the internet. There are two types of IP addresses in use: IP version 4 (IPv4) and IP version 6 (IPv6). IPv4 addresses are 32-bit numbers which are normally expressed as 4 octets in "dotted decimal" notation, such as 10.0.0.1. It was deployed in January 1983 and is the most commonly used version. IPv6 addresses are 128-bit numbers and are expressed as hexadecimal strings, such as 2001:0ab2:3b3::56. It started to be deployed in 1999. Although it was supposed to replace IPv4, it's full deployment still didn't happen. 

\begin{figure}[h!]
\centering
\includegraphics[scale=0.3]{IR-structure-diagram.png}
\caption{Global Structure of the Internet Registry System}
from http://www.ripe.net/internet-coordination/internet-governance/internet-technical-community/the-rir-system
\label{fig:ir_structure_diagram}
\end{figure}

IP addresses are assigned in hierarchical manner as shown in Figure \ref{fig:ir_structure_diagram}. Internet Service Providers obtain allocations of addresses from a Local Internet Registry (LIR), a National Internet Registry (NIR) or from the corresponding Regional Internet Registry (RIR) and in turn they assign IP addresses to end users. IANA's main purpose is to allocate IP addresses from pools of unallocated addresses to the Regional Internet Registries according to their needs. IANA does not make allocation of IP addresses directly to ISPs or users except for allocations of multicast addresses or protocol specific needs. If a RIR requires further IP addresses for allocation, IANA makes an additional allocation to the RIR. There are five RIRs, each one responsible for a specific region: AFRINIC for Africa Region, APNIC for Asia Pacific Region, ARIN for North America Region, LACNIC for Latin America and the Caribbean Regions, RIPE NCC for Europe, Middle East and Central Asia Regions. The five RIRs are represented in Figure \ref{fig:rirs_image}.

\begin{figure}[h!]
\centering
\includegraphics[scale=0.3]{RIR.jpeg}
\caption{Regional Internet Registries Service Regions}
from http://www.ripe.net/internet-coordination/internet-governance/internet-technical-community/the-rir-system
\label{fig:rirs_image}
\end{figure}

\subsection{Regional Internet Registries}
from http://www.ripe.net/internet-coordination/internet-governance/internet-technical-community/the-rir-system and http://www.ripe.net/lir-services/resource-management/faq/faq-ipv4-address-space


Among the tasks of a RIR are the coordination and representation of the members in its region. RIRs work together in order to develop consistent policies and promote best current practice for the Internet. As mentioned before RIRs are also responsible for allocating and assigning IP addresses within their respective regions. 

Local Internet Registries (LIR) are  established under the authority of a RIR and are responsible for the distribution and registration of address space at a local level. LIRs also ensure that resource allocation policies from RIRs are followed at a local level. LIRs are essentially members of a RIR and are normally operated by Internet Service Providers, but can also be large corportaions, governments and regulators.

from http://meetings.ripe.net/ripe-42/presentations/ripe42-lir-rfc2050/sld002.html using rfc - http://www.rfc-editor.org/rfc/rfc2050.txt and http://www.rfc-editor.org/rfc/rfc7020.txt

The Internet Registry IP Allocation Guidelines were described in RFC 2050, which became obsolete with the introduction of RFC 7020. The objective of this document is to provide information about the Internet Numbers Registry System used in the distribution of globally unique Internet Protocol (IP) address space and autonomous system (AS) numbers and not to propose changes to the current system. Internet number resources are distributed according to three main goals: Allocation Pool Management; Hierarchical Allocation and Registration Accuracy. The goal of Allocation Pool Management is to allocate number resources in accordance to the needs of those running networks, but takink into consideration pool limitations when allocating. This is relevant as the pools from which IP addresses and AS numbers are allocated are finite. Hierarchical Allocation refers to the need of allocating IP addresses in a way that permits their aggregation into a minimum number of routing announcements, in order to ensure continued scaling of the Internet's routing system. The last goal, Registration Accuracy,  requires that a registry of allocations is maintained in order to provide accurate registration information of allocations and ensure uniqueness. This way it is ensured that IP addresses and AS numbers are not allocated to more than one organization.  


   
from http://www.ripe.net/ripe/docs/ripe-606

\subsection{Border Gateway Protocol}

from paper bgpsurvey.pdf

The internet has a decentralized architecture composed of many interconnected networks, where the end systems are called hosts. At an intermediate level there are routers, which are responsible for selecting the paths that information takes in order to get to the respective end system. The selection of the right path is done by a routing process is determined by routing protocols. These protocols are responsible not only for performing path selectiong, but also to communicate reachability information. A group of IP networks which has a single defined external routing policy is called an Autonomous System (AS) and the process of routing whithin an AS is called intradomain routing, while routing between ASes is called interdomain routing. The de facto protocol used on the Internet for interdomain routing is the Border Gateway Protocol (BGP).
Although BGP has been deployed since the comercialization of the Internet, and version 4 of the protocol has been used for over a decade, it doesn't provide any kind of performance or security guarantees. This limition has already contributed to instabilities and outages \cite{Misdirection} such as a youtube outage caused by Pakistan Telekom \cite{Pakistan}. 

	Thousands ASes make part of the internet and use BGP to exchange information about how to reach blocks of destination IP addresses, which are called IP prefixes. Everytime a new route is available a BGP-speaking router sends an announcement message and when a route no longer exists, it sends a withdrawal message (incremental protocol). BGP is considered a path-vector protocol, because each AS adds its own AS number to the beginning of the AS path before advertising the route to the next AS. Each router will then select one preferred BGP route for each destination prefix. When selecting a route complex policies may be applied which will also decide whether to advertise the route to a neighboring router in another AS.   

	- IP prefixes and AS Numbers
	
	An IP address is a 32-bit number, normally represented in dotted-decimal notation (168.1.2.3), where each of the four octets is represented by a seperate integer. When addresses are assigned to institutions they are blocks of adjacent addresses, which will be represented by the first address of the block and a mask length. The prefix 168.1.2.0/24 contains all the addresses where the first three octets are 168.1.2 and the fourth octet is between 0 and 255, meaning all the addresses between 168.1.2.0 and 168.1.2.255. Allocating addresses in blocks also has the advantage that these blocks can be advertised by routers as a block, instead of having to advertise every IP address, leading to smaller routing tables and less route advertisements. IP prefixes may be contained within another, when that happens an IP router decides to forward a data packet by choosing the longest prefix match according to the destination IP address of that packet. For example, if a router has routing information for two prefixes 168.1.0.0/16 and 168.1.2.0/24 and needs to forward a packet with the destination IP address 168.1.2.3 it will choose the more specific prefix 168.1.2.0/24. 

AS numbers are assigned in a similar manner where from 1 to 64511 are public and have Internet-wide scope. Each number corresponds to only one AS. These numbers can appear in the AS-path attribute of BGP advertisements. Some institutions may not need an unique AS number, for example, if it connects to a single upstream provider that has the responsibility of providing connectivity to the rest of the Internet. In this case the customer may be assigned a private AS number with ranges from 64512 to 65535 for BGP communication with its provider. The provider would then advertise the routes on behalf of the customer. This allows providers to reuse the same private AS number for their customers.

When an AS introduces a destination prefix into the global routing system, by advertising the prefix to its neighboring ASes, it is called the originating AS. In Figure \ref{fig:bgpAdvertisement} we see AS 1 advertising a route for 10.0.0.0/8 with an AS path of "1". When AS 2 receives this advertisement, appends its own AS number to the front of the path and advertises the route with an AS path of "2,1" to the next AS . In the end of this process we see that AS 5 has two diferent paths to the route 10.0.0.0/8 related to its neighbouring ASes.

\begin{figure}[h!]
\centering
\includegraphics[scale=0.5]{bgpAdvertisement.png}
\caption{BGP Advertisement}
\label{fig:bgpAdvertisement}
\end{figure}

One vulnerability is that BGP does not ensure that a BGP-speaking router making the advertisement for on AS uses the AS number that this AS holds, neither ensures that the AS holds the prefixes it is advertising. As long as the neighboring router accepts the routes, a router can be configured to advertise routes into BGP with any AS number or for any destination prefix, including very small blocks (/30) or address blocks it does not hold. It is possible for the neighboring router to reject such cases, but it needs to be configured for that, meaning that a prior knowledge of acceptable prefixes and prefix lengths is needed. This makes the routing system quite vulnerable to misconfiguration or malicious attacks. The action of an AS advertising an unassigned prefix or belonging to another AS is know as prefix hijacking. When neighboring ASes receive this advertisement they might select the route and direct traffic towards this AS, as well as advertise this BGP route to their own neighbors. In Figure \ref{fig:bgpMaliciousAdvertisement} that AS 1 is advertising route 10.0.0.0/8 and it is the holder of this prefix. If AS 5 starts advertising the route to 10.0.0.0/8 and its neighbours select the shortest path routes, AS 3 and AS 4 will start diverting its traffic with 10.0.0.0/8 destination towards AS 5.

\begin{figure}[h!]
\centering
\includegraphics[scale=0.5]{bgpMaliciousAdvertisement.png}
\caption{BGP Malicious Advertisement}
\label{fig:bgpMaliciousAdvertisement}
\end{figure}

In the case that the offending AS just drops all packets destined to the hijacked addresses, the effect is known as a black hole, where the destinations seem unreachable to the parts of the Internet affected by this prefix advertisement. In the case that the AS decides to direct the traffic to hosts under its control the effect can be more severe, as the they can pretend to be the service provided by the hijacked destination. In this situation the traffic received by the AS can be analyzed and sensitive information, such as passwords or credit-card numbers, can fall in the wrong hands. It can also happen that the prefix hijacking is used to analyze the traffic before forwarding it to the correct destination, which would be a breach in the user's privacy. In order for an AS to do such an hijacking attack, it could advertise more specific prefixes then the ones in the original block (e.g., 10.1.128.0/17 and 10.1.0.0/16). This would work because of the longest prefix match rule used by IP routers, that forward packets to the more specific address range. In Figure \ref{fig:bgpDeaggregationAdvertisement} we can see an example. By advertising a more specific prefix (10.0.0.0/9), AS 5 will receive the traffic that had destination to AS 1 .

\begin{figure}[h!]
\centering
\includegraphics[scale=0.5]{bgpDeaggregationAdvertisement.png}
\caption{BGP Deaggregation Advertisement}
\label{fig:bgpDeaggregationAdvertisement}
\end{figure}


Routers exchange BGP announcement and withdrawal messages by establishing a BGP session with a pair. The BGP session runs over a TCP (Transmission Control Protocol) connection, which provides a reliable way to deliver a stream of ordered bytes, providing error correction and retransmission. BGP neighbors often have a direct physical link at the IP layer, but it can happen that they might have to communicate through an intermediate device. when a BGP session is extablished between ASes it is called an external BGP (eBGP) session, if the session occurs between routers in the same AS it is called an internal BGP (iBGP) session. This last type of sessions happen so that BGP routes learned from neighbors are spread throughout the AS. The communication between two BGP-speaking routers is also vulnerable to attacks, such as attacks against confidentiality, attacks against message integrity and denial-of-service attacks.

Besides being connected through physical links ASes are also bound by business or organizational relationships. An AS can serve as a provider to another organization and therefore there are contractual agreements, which are defined by service level agreements (SLAs). These agreements indicate the quality of service provided or define where the two ASes connect to each other and the traffic they will carry. For such reasons network operators need to be able to influenciate which BGP routes are chosen and which direct traffic will be accepted. For that they need to be able to specify routing policies. With BGP it is possible to enforce routing policies, such as forwarding data just for selected customers by using a number of protocol features. On of such features is the assignment of attribute values in UPDATE messages. BGP-speaking routers choose preferred routes by comparing the route attributes of the possible routes for each destination prefix. When network operators set specific fields in advance they can influenciate how route attributes are set. Among the most important BGP route attributes are:

- Local preference: It is used to override shortest-path routing by prefering other poliy goals and it is propagated inside an AS. It is normally used to prefer routes through a paying customer instead of other neighbors even if the path is longer. This is achieved by assigning a higher local preference value to the router's next-hop AS of the preferred route then to the other next-hops. The same can be applied to direct traffic towards less overcrowded connections, by assigning a high local preference value to this route. 

- AS path length: As mentioned before BGP is a path vector algortihm because each AS adds its AS number to the path before advertising the route. If several routes have the same local-preference value, the route with the smallest AS-path length is chosen, therefore an AS can increase the length of the AS path by adding its AS number to the path multiple times, resulting in a less attractive route to other ASes. This process is known as AS prepending.

- Origin type: Another way to choose a route in case of a tie between several ones is by checking the origin type. A route learned within the AS can be preferred to one learned from the outside, allowing an AS to modify the origin-type attribute to influentiate the choice of a route.

- Multi-Exit Discriminator (MED): It is possible that two neighboring ASes connect to each other at several geographic locations. The MED attribute is used to ensure that traffic is sent to the other AS through the peering location route with the smaller MED. This attribute is typically specified as part of the contract betwwen the ASes.


As seen before BGP routers can can be configured with route attributes in order to influentiate the route that is chosen. This allows to filter received (import policy) and advertised routes (export policy) to its neighbors while choosing where to forward its traffic. The problem is that the way an AS selects routes can be manipulated by sending BGP route announcements with bogus attributes. The AS-path attribute could be truncated to look shorter and therefore more attractive or add additional AS hops at the end. An AS could remove a particular hop from the AS path to modify traffic through certain ASes, or may add an AS number to the AS path so the the target AS would delete ith own AS number thinking it was a loop. An AS could also try to attach MED values to the routes to try to influentiate the route decision. 

from book computer networking

BGP provides each AS ways  to:

- Obtain reachability information from neighboring ASes.
- Propagate the reachability information to all routers internal to the AS
- Determine "good" routes to subnets based on the reachability information and AS policy
- Most importantly, BGP allows each subnet to advertise its existence to the rest of the Internet

In BGP, destinations are not hosts bu CIDRized prefixes, with each prefix representing a subnet or a collection of subnets. BGP peers advertise routes to each other. Two important attributes are AS-PATH and NEXT-HOP:
 - AS-PATH: contains the ASs through which the advertisement for the prefix has passed. When a prefix is passed into an AS, the AS adds its AS number to the AS-PATH attribute.
 -NEXT-HOP: is the router interface that begins the AS-PATH. 



D. Routing Registries
Despite the benefits of protective route filtering,
detecting and disregarding bogus BGP routes is more
challenging when the erroneous information stems from a
misconfiguration or an attack several AS hops away.
Having a shared, global view of Bcorrect[ routing
information would make it much easier to detect invalid
routes. An accurate routing registry [48] of, for example,
prefix ownership, AS-level connectivity, and routing
policies would enable security-conscious ASes to detect
and discard invalid routes. 


\subsection{Internet Routing Registries}
http://www.nanog.org/meetings/nanog51/presentations/Sunday/NANOG51.Talk34.NANOG51%20IRR%20Tutorial.pdf

from http://www.nsfnet-legacy.org/about.php
http://www.nsf.gov/about/history/nsf0050/internet/launch.htm
http://www.hernandocadett.com/content/stability-and-consistency-internet-wide-network-routing
https://www.nsf.gov/news/news\_summ.jsp?cntn\_id=103050
http://www.merit.edu/research/nsfnet\_article.php
http://www.merit.edu/services/meritradb/history.php

The National Science Foundation inherited the responsability for developing the U.S. Internet from the Advanced Research Projects Agency (ARPA) and the Internet Routing Registry concept goes back to the 1980's and to the National Science Foundation Network (NSFNet).  NSFNET refers to the program sponsored by the National Science Foundation, initiated in 1985, with the purpose of support and promote advanced networking among U.S. research and education institutions. Among its participants were Merit Network, Inc., IBM, MCI, Advanced Network \& Services, Inc., the State of Michigan and many institutions in research and education. As a way to configure the MSFNET's backbone routers, the Policy Routing Database (PRDB) was used since 1989. Merit planned the retirement of PRDB for December 1994, which would be replace by the Routing Arbiter Database (RADB). RADB would then become part of the Internet Routing Registry (IRR) along with RIPE NCC, MCI and other registries. The IRR objective was to be a global public repository of announced routes and routing policy in a common format so that ISPs could use this information to configure their backbone routers and analyze routing policy.

\begin{figure}[h!]
\centering
\includegraphics[scale=0.5]{nsfnet_topology.png}
\caption{NSFNET Topology}
from http://www.merit.edu/research/nsfnet.php
\label{fig:nfsnet_image}
\end{figure}

PRDB goal was to maintain information regarding legitimate destination announcements from the various regional networks. This information was very important in order to prevent routing loops. Only after BGP replaced EGP as the inter-domain routing protocol, it became no longer necessary to controll the avoidance of routing loops in such a administratively way. With this change, the information in PRDB turned to be mainly used to gather routing policies, such as path preferences and to generate configuration files for backbone routers.

With the transition of NSFNET to the commercial Internet, the National Science Foundation selected Merit Network and a partner organization to act as Routing Arbiters. The goal of the PRDB would be now to record global routing policy information based on each Autonomous System's policy and RADB comes into the scene. RIPE had pioneered work to record global routing policy information in Europe and the data exchange format described in RIPE-181 (RFC 1786) was adopted as the standar for Internet Routing Registries. This model was also taken by RADB. 

The transition from PRDB presented several problems as the tools used to configure the NSFNET/ANSnet routers were based on PRDB attributes and not RIPE-181. But by December 1994, all the data in PRDB had been converted to RIPE-181-style expressions and on February next year, the RADB had been populated with RIPE-181-style Maintainer and AS Objects. 

\subsubsection{RIPE-181}
from http://www.irr.net/docs/rfc1786.txt


The RFC-1786, which was originally published as a RIPE document (RIPE-181), describes the original database formats that were used by RIPE NCC to store routing policy in its database. As mentioned before RIPE also serves as an allocation registry, which means that its database also contains non-routing oriented objects. 
This documents was also referred as ripe-81++ as it was an update to the original 'ripe-81' proposal for representing and storing routing polices in the RIPE database. Several extensions proposed by Merit Inc. were incorporated into this document and it provided a generalized IP routing policy representation to be used by the Internet routing registries. As it's purpose is to be a general document for Internet routing registries, one can replace the RIPE routing registries by "Regional routing registry".

This document is an important source of information to understand today's Internet registries. 
The Regional routing registries database contains both routing registry and address space allocation registry information. In the begining both informations were combined, but with later it became clear that a separation
of routing information and allocation is desirable. Mainly because in some parts of the world there are different registries for each kind of information and also because often the maintainer of the routing information is not the same as the one of the allocation information.

One of the activities of routing registries is to maintan a databse of IP networks, DNS domains, with information of contact persons and other network management information. The content of this database can be queried using the whois protocol or retrieved as a whole.
The allocation registry contains data about address space allocated to enterprises or delegated to local registries as well as data about the domain name space. 
In the Regional routing registries database the information is stored in form of objects. The types of objects are summarized in table \ref{table:1}.   
   
\begin{table}[h!]
\centering
\begin{tabular}{ | c | c | l | }
\hline
 Registry & Object & Describes \\ \hline
 B & person & contact persons \\
 A & inetnum & IP address space \\
 A & domain & DNS domain \\
 R & auto-num & autonomous system \\
 R & as-macro & a group of autonomous systems \\
 R & community & community \\
 R & route & a route being announced \\
 R & clns & CLNS address space and routing \\
 \hline
\end{tabular}
\caption{Summary of database objects in routing registries}
\label{table:1}
\end{table}

The Registry column gives information to which registry the object belongs to, "A" for allocation registry, "R" for routing registry and "B" for both.

The Objects are represented by attributes value pairs. 
An example of an whois query to retrieve information about network 192.87.45.0 is shown in table \ref{table:2}. Here we can see one inetnum object and two person objects.


\begin{table}[h!]
\centering
\begin{tabular}{  r  l  }
 
 inetnum:  & 	192.87.45.0 - 192.87.45.255 \\
 netname:  & 	OCLC-NET \\
 descr:    &	OCLC \\
 country:  & 	NL \\
 admin-c:  & 	WK1844-RIPE \\
 admin-c:  & 	KDB18-RIPE \\
 tech-c:   & 	WK1844-RIPE \\
 tech-c:   & 	KDB18-RIPE \\
 status:   & 	LEGACY \\
 remarks:  & 	For information on "status:" attribute read https://www.ripe.net/data-tools/db/faq/faq-status-values-legacy-resources \\
 mnt-by:   & 	SN-LIR-MNT \\
 mnt-irt:  & 	irt-SURFcert \\
 source:   & 	RIPE \# Filtered \\
 \\
 person:   & 	Kees van Dobben de Bruyn \\
 address:  & 	Pica Centrum voor Bibliotheekautomatisering \\
 address:  & 	P.O. Box 876 \\
 address:  & 	NL - 2300 AW Leiden \\
 address:  & 	The Netherlands \\
 phone:    & 	+31 71 257174 \\
 fax-no:   & 	+31 71 223119 \\
 nic-hdl:  & 	KDB18-RIPE \\
 mnt-by:   &  	SN-LIR-MNT   \\
 source:   & 	RIPE \# Filtered \\
 \\
 person:   &    Wim Kooreman\\
 address:  &    Pica Centrum voor Bibliotheekautomatisering\\
 address:  &    P.O. Box 876\\
 address:  &    NL - 2300 AW Leiden\\
 address:  &    The Netherlands\\
 phone:    &    +31 71 257257\\
 fax-no:   &    +31 71 223119\\
 nic-hdl:  &    WK1844-RIPE\\
 mnt-by:   &    SN-LIR-MNT\\
 source:   &    RIPE \# Filtered\\
 
\end{tabular}
\caption{Example of whois query response to retrive information about network 192.87.45.0}
\label{table:2}
\end{table}     
        
 
Routing Registry Objects

The most important objects regarding the routing registry are the "aut-num" and the "route" objects. The "aut-num" object describes an autonomous system and the "route" object the route. The "auto-num" object provides contact information for the referred AS and describes its routing policy by identifying the neighboring ASes with which routing information is exchanged. The routing policy is described by identifying what is being announced and what is allowed. The "auto-num" objects provide information how routing information is propagated.
The "route" object describes a single route that is being injected and references the AS that is originating it. In table \ref{table:3} it is shown a "route" object returned from a whois query on network 192.87.45.0. 
The value of the route attribute is a classless address and represents the route being injected into the routing system. The value of the origin attribute is an AS referring to an "aut-num" object, which is the AS injecting this route.
  
\begin{table}[h!]
\centering
\begin{tabular}{  r  l  }

route:    &      192.87.0.0/16\\
descr:    &      SURFnet CIDR Block IV\\
origin:   &      AS1103\\
mnt-by:   &      AS1103-MNT\\
mnt-lower:&      SN-LIR-MNT\\
source:   &      RIPE \# Filtered\\

\end{tabular}
\caption{Example of whois query response to retrive information about network 192.87.45.0 "route" object}
\label{table:3}
\end{table} 


The Autonomous System Object

An Autonomous System (AS) is defined by a group of IP networks which has a single defined external routing policy. An AS has an unique number associated with it. This number is used to identify the AS and to exchange routing information. Routing protocols such as BGP and EGP are used to exchange routing information between ASes.
There are some recommendations that the creation and allocation of an AS should follow such as:
	- It is only needed to create an AS when exchanging routing information 	with other Ases.
	- In a case of customer networks connect to one service provider, the 		IP network should normally be a member of the service providers AS.
    - In the case that a network operator connects to more than one AS with 	different routing policies, it is required for them to have their own 		AS number. 
    - The AS should always try to be populated with as many routes as 		  possible, as long as all routes have the same routing policy
   
An As is represented by an "aut-num" object and "route" objects representing the routes originated by the AS. The "aut-num" object stores administrative information as well as the routing policies of the AS. The origin attributes of the route objects define the set of routes originated by the AS, where each object has only one origin attribute. In table \ref{table:4} it is shown an example of a AS object from a whois query on AS1104.  
    
\# Use this object already new or the old example from the RFC???

\begin{table}[h!]
\centering
\begin{tabular}{  r  l  }

aut-num:    &      	AS1104\\
as-name:    &      	Nikhef\\
descr:   	&     	FOM-Nikhef\\
descr:   	&    	Science Park 105\\
descr:		&   	Amsterdam, 1098 XG\\
descr:   	&	    The Netherlands\\
import:  	&       from AS1103 accept AS1103\\
import:  	&       from AS1139 accept AS1139\\
import:  	&       from AS1888 accept AS1888\\
import:  	&       from AS1126 accept AS1126\\
import:  	&       from AS1124 accept AS1124\\
export:  	&       to AS1103 announce AS1104\\
export:  	&       to AS1139 announce AS1104\\
export:  	&       to AS1888 announce AS1104\\
export:  	&       to AS1126 announce AS1104\\
export:  	&       to AS1124 announce AS1104\\
admin-c: 	&       PK8221-RIPE\\
tech-c:  	&       PK8221-RIPE\\
remarks: 	&       For information on "status:" attribute read https://www.ripe.net/data-tools/db/faq/faq-status-values-legacy-resources\\
status:  	&       LEGACY\\
mnt-by:  	&       AS1104-MNT\\
source:  	&       RIPE \# Filtered\\

\end{tabular}
\caption{Example of whois query response to retrive information about AS1104}
\label{table:4}
\end{table} 

This representation provides a set of routes and a description of administrative details and routing policies. 
        
        
          
          
          
          


	See Appendix A for a complete syntax definition of the "aut-num"
   object.


   It should be noted that this representation provides two things:

       + a set of routes.

       + a description of administrative details and routing policies.

   The set of routes can be used to generate network list based
   configuration information as well as configuration information for
   exterior routing protocols knowing about ASes. This means an AS can
   be defined and is useful even if it does not use routing protocols
   which know about the AS concept.

   Description of routing policies between ASs with multiple connections
   - "interas-in/interas-out"

   The following section is only relevant for ASes which use different
   policies on multiple links to the same neighboring AS. Readers not
   doing this may want to skip this section.

   Description of multiple connections between ASs defines how two ASs
   have chosen to set different policies for the use of each or some of
   the connections between the ASs.  This description is necessary only
   if the ASs are connected in more than one way and the routing policy
   and differs at these two connections.

\section{Problem Statement}

from paper THE INTERNET IN TRANSITION: THE STATE OF THE TRANSITION TO IPV6 IN
TODAY'S INTERNET AND OF MEASURES TO SUPPORT THE CONTINUED USE OF
IPV4

As early as 1990, concerns were raised in the technical community that
certain aspects of this technology would not readily scale into ubiquitous global deployment in a world that
contained dense concentrations of communicating devices. In particular it was felt that the scope of
4 billion addresses defined within this communications protocol was simply insufficient for such a role.
The technical response to these concerns was to define a new version of the Internet Protocol that would
encompass a far larger address domain, namely version 6 of the Internet Protocol, or IPv6. The technical
specification of IPv6 was largely completed by 1996. At this time the technical community initiated a
programme to promote the need to transition the Internet to use IPv6, and proposed a deadline for
completion of this transition well in advance of the anticipated exhaustion of the IPv4 address pool. The
transition process had one major challenge, in that IPv6 is not backward compatible with IPv4.This
implies that this transition is not a simple case of replacing IPv4 with IPv6 and moving on. Networks, edge
devices, and service delivery points all need to support operation in both protocols simultaneously, and do
so until the entire Internet has this dual protocol capability. Only then can the IPv4 component of the
network be phased out of service.

As this report illustrates with various measurements, the transition to IPv6 is still in its early
phase. Meanwhile, the Internet has continued with accelerating growth, as each year sees greater levels of
deployment of connected devices than previous years. The initial wave of expansion of the wired access
network has been complemented in recent years by the co-opting of mobile telephony services by the
hand-held Internet device. This expansion of the scope of the network places further pressures on the
dwindling pool of available IPv4 addresses, to such an extent that the ongoing supply of IPv4 addresses
that has fuelled the continued growth of the Internet has now ceased in Europe, the Middle East, Asia and
Oceania. At the same time the transition to IPv6, through a dual-stack capable Internet, has not followed
quite the same path of widespread rapid deployment.

Some parts of the industry have appreciated the nature of the issues relating to running out of
addresses within the network's infrastructure, and have planned and acted accordingly. Microsoft's
widespread Windows products have included an IPv6 protocol engine alongside their IPv4 engine in their
operating system since late 2001. Similar moves have been made by Apple in their MAC OSX operating
system, and support is also to be found in Linux systems, and Linux-based derivative platforms, such as
Android. Much of the infrastructure of the network, including backbone transmission paths and the
Domain Name System (DNS) infrastructure, is also IPv6-capable. Not all parts of the industry, however,
have interpreted this message about the need for transition to IPv6 as an imperative call for action. The
result is that while it appears that over one half of the end-user equipment deployed on the wired Internet is
capable of supporting IPv6 today, less than 2 percent of this same equipment is able to support connections to
external services using IPv6.1


All these providers have recently made significant efforts to extend IPv6 access to consumers through their
mass-market Internet access services. However, IPv4 address depletion has imposed a further agenda on
many service providers, many of whom now appear to be concentrating their current efforts on the
deployment of technologies that will conserve their remaining IPv4 address stocks by deploying address-
sharing middleware that will permit the sharing of IPv4 addresses across multiple customers. This potential
change in the immediate agenda for many entities in this sector may have profound implications for the
future of the Internet.

Introduction from paper

The expectation at the time was that ISPs would react prudently to news of the impeding
exhaustion of the IPv4 address pool and would collectively switch over to an all-IPv6 network well before
the remaining pool of available IPv4 addresses was depleted. Unfortunately, the technical and economic
incentives-related challenges associated with converting large numbers of interconnected IP networks to
IPv6 were not adequately appreciated. As a result, this has not been the case. The central IPv4 address
pool, managed by the operator of the Internet Assigned Numbers Authority (IANA) was fully allocated in
February 2011, and, by August 2013, two of the five Regional Internet Registries had reached critical low
points in their locally managed IPv4 address pools. The Internet, however, continues to rely on IPv4 for the overall majority of end users and services, and the visible levels of IPv6 deployment remain extremely
low.

For this reason the transition to IPv6 was planned as a dual-stack transition, where the initial
phase of the transition would see a piecemeal change of parts of the Internet from exclusively using IPv4 to a dual protocol stack mode that supports both IPv6 and IPv4.

Addresses and address structures
Addresses in this context are defined by a communications protocol. The address is a unique
identifier used by the communications protocol to distinguish active elements, so that the protocol can
provide the necessary support to enable the communication of data between endpoints of the network
defined by the operation of the protocol.
In the context of the Internet protocol an "IP address" is a unique identifier assigned to a
computer's interface that runs the internet protocol suite (IP) and is connected to an IP network. The
common format of an IP address is a 32-bit number, allowing for a theoretical maximum of some
4.3 billion values for unique addresses.
The IP address has a minimal internal structure, which is a division into a network part and a host
part. All hosts connected to a common network share the same value of the network part of their address,
and uniquely identify themselves by the unique host part of the address.

Statistics on address distribution
The IANA functions operator handed out its last IPv4 address block in February 2011. Some
592,708,864 addresses remain registered with the IANA functions operator as "IETF Reserved" address
blocks. A further 20,466,432 addresses are being held by the IANA functions operator, representing
returns to the registries of address blocks that were originally allocated to address holders prior to the
introduction of the RIR system. These addresses will be evenly distributed back to the RIRs in the future in
accordance with a global policy adopted in all the regions of the RIR system and ratified by ICANN as the
IANA functions operator.14

Table 1. IPv4 address holdings by regional internet registry (August 2013)
Source: Regional Internet Registries Extended Statistics Reports (www.nro.net/pub/stats/nro/delegated-extended)

In the case of APNIC and the RIPE NCC, the level of available IPv4 address space is now below
that of a /8 (16,777,216 addresses). According to the address distribution policies adopted in both these
regions, the address allocation policy now permits a maximum allocation of 1,024 addresses to any single
applicant.

"Used" and "unused" addresses
When addresses are "assigned" to a local registry or ISP, the addresses can be used in a number
of ways. For an ISP, or a data centre operator, it may well be the case that the address is directly used to
support interactions over the public Internet. If this is the case then it is necessary that reachability to this
address be advertised on the Internet's routing system, so the network will be announced into the Internet's
routing system. Each individual routing announcement covers a span of addresses, so it is not necessarily
the case that each and every individual address covered in the network span is uniquely assigned to an
individual end device, but overall the announcement of a network address prefix in the Internet's routing
system is commonly interpreted as a strong indicator that the addresses in that span are being "used" in
some fashion.
The corollary is that if a network block is not announced to the global routing system, then this
may be interpreted as an indication that the address block is no longer in use. This is not necessarily the
case, as addresses are assigned for various purposes, including, but not necessarily limited to, their use in
the public Internet. So the lack of an announcement in the Internet's routing system is not necessarily a
clear indication that the address is no longer being used in an active network.

The address policy discussions within each of the regions has, from time to time, discussed the
possibility of attempting to reclaim and reuse those addresses that are not in use. This option presents some issues for the registries, in so far as the agreements between the registries and address holders do not necessarily commit the address holders to immediately advertise those addresses on the public Internet, and various forms of use of addresses in contexts where the address is not advertised on the public Internet are considered acceptable forms of use. It is also the case that a significant volume of addresses were distributed by the predecessors of the RIR system, and the arrangements in place at that time placed little in the way of formal obligation on the address holder. The ability of an RIR to effectively resume such addresses without any form of active consent on the part of the address holder is generally considered to be beyond the acknowledged powers of the registry. Address resumption efforts have had more success in
those contexts where there is an active agreement between the address holder and the registry, and where
the address holder has not fulfilled their obligations with respect to this agreement. However the total
volume of addresses that the registries have been able to reclaim from such lapses of these address holding
agreements have not been all that significant, and have had no significant impact on the address runout
experience in the case of APNIC and the RIPE NCC, or in the runout projections relating to ARIN,
AFRINIC and LACNIC.

Table 3. IPv4 Advertised and unadvertised address holdings by regional Internet registry (August 2013)

Predictions on IPv4 runout
The central pool of IPv4 addresses managed by the IANA functions operator distributed its final
set of address allocations to the RIRs early February 2011. At this stage it continues to manage a set of
address reservations made by the IETF as part of its protocol parameter registry services function, and also
manages a temporary pool of returned so-called legacy addresses prior to their re-distribution to the RIRs.
Of the five RIRs, the first to reach a conclusion of the general address allocation function for
IPv4 addresses was APNIC, on 19 April 2011. Since then, APNIC is operating its continuing IPv4
allocations under a "Last /8 Policy" where each serviced entity may apply for one, and only one, allocation
of up to 1024 addresses. The intent of this policy is to hold onto a small pool of addresses to assist new
entrants in the area of Internet Service Provision to operate in dual-stack mode with some small amount of
IPv4 to service the IPv4 side of their dual-stack needs, with the explicit awareness, noted when the Asia
Pacific regional address policy community was contemplating the adoption of this particular address
policy, that the address block available to each applicant under this policy could be used in conjunction
with IPv4 CGNs so as to allow this very small block of IPv4 addresses to be used in far larger dual-stack
networked environments.15
The second RIR to also run to the end of its general address allocation policies has been the RIPE
NCC, which exhausted its pool on 14 September 2012. The RIPE NCC has also moved into a framework
of a Last /8 Policy with similar constraints in place in APNIC.16
The remaining three RIRs, namely ARIN, LACNIC and AFRINIC, are yet to run out of IPv4
addresses.

In the case of the ARIN registry, it currently holds some 31,267,072 addresses in its local address
pool. In estimating the projected run-out time for this registry it is noted that the model of address
consumption in the region served by this registry changed significantly at the same time as the IANA
registry was depleted in February 2011, which was also the time when this registry changed from a policy
framework that assigned addresses to entities in a quantity that encompassed their planned requirements
for the forthcoming 12 months to one that encompassed only three months of future requirement.


IPv6

The transition process being used for the deployment of IPv6 in the Internet is a so-called
"dual-stack" transition. What is possible is to support the various upgrades of the components of a network
from IPv4-only to a dual-stack network that supports the active use of IPv4 and IPv6 concurrently. This
mode of operation is a "dual-stack" mode, where devices that support both protocol stacks have the choice
be being able to use either protocol.


from paper Scarcity in IP addresses: IPv4 Address Transfer Markets and the Regional Internet Address Registries http://www.internetgovernance.org/wordpress/wp-content/uploads/IPAddress\_TransferMarkets.pdf

What should be done? This paper evaluates transitional policies that Internet
governance agencies are considering as a response to the depletion of the IPv4 address
space. In particular, the paper focuses on proposals to allow organizations holding IPv4
addresses to sell address blocks to other organizations willing to buy them. IP address
transfer markets, as they are called, have been proposed as a pragmatic way to extend
the life of the IP address space. One important benefit of such a policy is to provide
incentives for existing holders of addresses to release unused address resources.
Another possible benefit is the way it might rationalize and make more transparent an
underground economy in address resources. Transfer markets also increase the
autonomy of Internet users by providing an alternative to the centralized administrative
processes that currently control address allocations. 

This paper analyzes the economics of the proposed transfer policies, and
conducts a systematic comparison of the policies proposed in the three main world
Internet regions. It concludes that:
• Address transfer markets offer a pragmatic solution to the problem of reclaiming
a substantial amount of unused IP address space and of re-allocating addresses to
their efficient uses
• The risks of instituting well-designed address transfer policies are small when
compared to the potential benefits. The change is less radical than it appears.
• A failure to legitimize address transfer markets would create substantial risks of
the institutionalization of gray or black markets in IPv4 address resources, leading
to a deterioration of accurate registration and administration of the legacy address
space. This could have severe negative implications for Internet security.
• IPv4 address transfers should not be prevented as part of an attempt to push
organizations into IPv6. If migration to IPv6 is beneficial, a transfer market could
only prolong, not stop, the transition. We do not know how long it will take the
global Internet to transition to IPv6, or even whether such a migration will
succeed. Given these uncertainties, a longer transition period may turn out to be
very helpful.
• The proposed address transfer policies being considered by RIPE and APNIC are
more liberal than ARIN’s. All three could be improved in various ways. Most of the
legacy IPv4 address space is in North America; thus, the policies ARIN adopts have
the most importance and should be formulated with the good of the global
Internet in mind. RIPE, ARIN and APNIC should strive to harmonize their transfer
policies and over the longer term make inter-regional transfers possible.


from paper Dimensioning the Elephant: An Empirical Analysis of the IPv4 Number Market http://www.internetgovernance.org/wordpress/wp-content/uploads/IPv4marketTPRC20122.pdf

After a highly publicized deal in which Microsoft bought Nortel’s number assets in a bankruptcy proceeding, the reality of an IP number market can no longer be denied. But many in the Internet technical community still feel uncomfortable about it. One reporter with attitudes typical of the technical community, has predicted that “a functioning market won't form at all, or will break down very quickly after it forms" (van Beijnum, 2011). This is partly due to an ideological resistance to the commodification of a critical Internet resource, and partly due to their fears that an IPv4 market might delay or even prevent a migration to a new Internet protocol (IPv6). Either way, few wish to openly acknowledge the market’s existence. Thus the topic of number markets brings to mind the phrase “the elephant in the room.” 

Moreover, the information we have about this elephant is fragmented and unsystematic. IP number allocation is controlled on a contractual basis by five separate regional Internet registries (RIRs). Each has different policies toward transfers, different registry databases, and different disclosure practices. It is difficult, therefore, to obtain a comprehensive picture of the emerging market for IPv4 number blocks. This recalls the old fable about the five blind men and the elephant, with each one having access to a small part of the body and none of them quite grasping the nature of the beast as a whole.



IPv4 scarcity 

\section{Related Work}
- Capturing Ghosts: Predicting the Used IPv4 Space by inferring Unobserved Addresses
- A first llok at IPv4 transfer Markets
- A comparative study on IP Prefixe and their Origin ASes in BGP and the IRR
- The Great IPv4 Land Grab: Resource Certification for the IPv4 Grey Market
- Evolution of Internet Address Space Deaggregation: Myths and Reality


\section{Approach}
 - RouteViews collectors why? Why is it enough? Why there is no need for RIPE collectors?
 - Whois
 - RADB
\section{Results}
\section{Conclusion}
\bibliographystyle{plain}
\renewcommand\bibname{\section{References}}
\begin{thebibliography}{count}

\bibitem{Misdirection}
    Rensys Blog,
    \emph{The New Threat: Targeted Internet Traffic Misdirection}.
    [Online]. Available: http://www.renesys.com/2013/11/mitm-internet-hijacking/


\bibitem{Pakistan}
    Rensys Blog,
    \emph{Pakistan Hijacks YouTube}.
    [Online]. Available: http://www.renesys.com/blog/2008/02/pakistan\_hijacks\_
youtube\_1.shtml

\bibitem{Address_Space_Deaggregation}
	L. Cittadini, W. Mühlbauer, S. Uhlig, R. Bush, P. Francois and O.Maennel,
	\emph{Evolution of Internet Address Space Deaggregation: Myths and Reality}.
	IEEE JSAC, Aug 2010.
	
\bibitem{IANA_Address_Space}
	IANA,
	\emph{IANA IPv4 Address Space Registry}.
	[Online]. Available: http://www.iana.org/assignments/ipv4-address-space/ipv4-address-space.xhtml

\end{thebibliography}	
\end{document}